\documentclass[11pt]{article}

\usepackage[margin=0.5in,top=0.8in,bottom=0.8in]{geometry}
\usepackage{amsmath,amssymb}

\newcommand{\beq}{\begin{equation}}
\newcommand{\eeq}{\end{equation}}
\newcommand{\beqn}{\begin{eqnarray}}
\newcommand{\eeqn}{\end{eqnarray}}
\newcommand{\ve}[1]{\mbox{\boldmath $#1$}}
\newcommand{\la}{\stackrel{<}{ _{\sim}}}

\numberwithin{equation}{section}

\begin{document}

\title{Mathematics Involving Non-Parametric Statistics}
\date{2016-05-04, updated on 2017-11-15}
\author{Yuk Tung Liu}
\maketitle

\bigskip

These short notes provide mathematical proofs of the following properties involving non-parametric 
statistics covered in Stat 200. 

{\bf Wilcoxon Rank Sum}: When there are no ties in both groups A and B, the expected value and 
variance of the rank sum for group A are 
\beq
  E(R_A) = \frac{n_A (N+1)}{2} \ \ \ , \ \ \ V(R_A) = \frac{n_A n_B (N+1)}{12} ,
\eeq
where $n_A$ and $n_B$ are the number of observations in group A and B, respectively. The total 
number of observations $N=n_A+n_B$. 
 
{\bf Relationship Between Wilcoxon Rank Sum and U Statistic}: The Wilcoxon Rank Sum for group A, 
$R_A$, is related to the U statistic for group A, $U_A$, by 
\beq
  R_A = U_A + \frac{n_A(n_A+1)}{2} .
\eeq

{\bf Spearman's Rank-Order Correlation Coefficient}: Suppose $x=(x_1, x_2, x_3, \cdots, x_n)$ and 
$y=(y_1,y_2,\cdots, y_n)$. If there are no ties in both $x$ and $y$, and $x$ and 
$y$ are uncorrelated, the expected value and variance of Spearman's rank-order correlation coefficient 
$r_s$ are 
\beq
  E(r_s) = 0 \ \ \ , \ \ \ V(r_s) = \frac{1}{\sqrt{n-1}} .
\eeq

\section{Wilcoxon Rank Sum}

To calculate the mean and variance of the rank sum $R_A$, we need to calculate several 
quantities.

We first want to calculate the values of two series: $1+2+3+\cdots + n$ and 
$1^2 + 2^2 + \cdots + n^2$. The first one is an arithmetic series, which can be 
computed as follows. 
\[
  S_n = 1+2+3+\cdots n 
\]
We can rewrite $S_n$ as 
\[
  S_n = n + (n-1) + (n-2) + \cdots + 1 
\]
Adding the two expressions gives 
\[
  2 S_n = \underbrace{(n+1) + (n+1) + \cdots + (n+1)}_{n\ {\rm times}} = n(n+1)
\]
Hence, 
\beq
  1+2+3+\cdots + n = \sum_{i=1}^n i = \frac{n(n+1)}{2} .
\label{eq:arithmetic}
\eeq
The second series is a discrete version of the integral 
\[
  \int_1^n x^2 dx 
\]
To calculate the sum, we consider the integral 
\[
  \int_{i-\frac{1}{2}}^{i+\frac{1}{2}} x^2 dx = \frac{1}{3} \left[ \left(i + \frac{1}{2}\right)^3 
- \left(i - \frac{1}{2}\right)^3 \right]  
\]
Now, 
\beqn
  \frac{1}{3} \left[ \left(i + \frac{1}{2}\right)^3 
- \left(i - \frac{1}{2}\right)^3 \right] &=& \frac{1}{3} \left[ \left( i^3 + \frac{3}{2}i^2 
+ \frac{3}{4}i + \frac{1}{8}\right) - \left( i^3 - \frac{3}{2}i^2 + \frac{3}{4}i - \frac{1}{8}\right) \right] 
\cr \cr 
&=& i^2+ \frac{1}{12} \nonumber
\eeqn
\[
 \Rightarrow \ \ \ i^2 = -\frac{1}{12} + \frac{1}{3} \left[ \left(i + \frac{1}{2}\right)^3
- \left(i - \frac{1}{2}\right)^3 \right]
\]
\beqn
  \Rightarrow \sum_{i=1}^n i^2 &=& -\frac{n}{12} + \frac{1}{3} \sum_{i=1}^n 
\left[ \left(i + \frac{1}{2}\right)^3 - \left(i - \frac{1}{2}\right)^3 \right] \cr \cr 
&=& -\frac{n}{12} + \frac{1}{3} \sum_{i=1}^n \left(i + \frac{1}{2}\right)^3 
- \frac{1}{3} \sum_{i=1}^n \left(i - \frac{1}{2}\right)^3 \cr \cr 
&=& -\frac{n}{12} + \frac{1}{3} \sum_{i=1}^n \left(i + \frac{1}{2}\right)^3 
- \frac{1}{3} \left(\frac{1}{2}\right)^3 - \frac{1}{3} \sum_{i=2}^n \left(i - \frac{1}{2}\right)^3 \cr \cr
&=& -\frac{n}{12} + \frac{1}{3} \sum_{i=1}^n \left(i + \frac{1}{2}\right)^3 
- \frac{1}{24} - \frac{1}{3} \sum_{i=1}^{n-1} \left(i + \frac{1}{2}\right)^3 \cr \cr 
&=& -\frac{n}{12} + \frac{1}{3} \left(n+\frac{1}{2}\right)^3 - \frac{1}{24} \cr \cr 
&=& -\frac{n}{12} + \frac{1}{3} \left(n^3 + \frac{3}{2}n^2 + \frac{3}{4}n + \frac{1}{8}\right) - \frac{1}{24} \cr \cr
&=& \frac{1}{3}n^3 + \frac{1}{2}n^2 + \frac{1}{6}n \cr \cr 
&=& \frac{n(2n^2+3n+1)}{6} \cr \cr 
&=& \frac{n(n+1)(2n+1)}{6} 
\label{eq:squaresum}
\eeqn

Let $x =$ permutation of $(1,2,3,\cdots, n)$. The expected value of $x_i$ is 
\beq
  E(x_i) =\frac{1}{n} \sum_{i=1}^n i = \frac{1}{n} \frac{n(n+1)}{2} = \frac{n+1}{2} 
\label{eq:Expxi}
\eeq

The variance of $x_i$ is 
\beqn
  V(x_i) &=& E(x_i^2) - E^2(x_i) \cr \cr 
&=& \frac{1}{n} \sum_{i=1}^n i^2 - \left( \frac{n+1}{2}\right)^2 \cr \cr 
&=& \frac{(n+1)(2n+1)}{6} - \frac{(n+1)^2}{4} \cr \cr 
&=& \frac{n^2-1}{12} ,
\label{eq:Vxi}
\eeqn
where we have used~(\ref{eq:squaresum}) in the third line.

The covariance $cov(x_i,x_j)$ for $i \neq j$ can be computed as follows.
\beqn
  cov(x_i,x_j) &=& E(x_i x_j) - E(x_i) E(x_j) \cr \cr 
&=& \frac{1}{n(n-1)} \sum_{i \neq j} ij - \left( \frac{n+1}{2}\right)^2 \cr \cr 
&=& \frac{1}{n(n-1)} \sum_{i,j} ij - \frac{1}{n(n-1)} \sum_{i=1}^n i^2 - \frac{(n+1)^2}{4} \cr \cr 
&=& \frac{1}{n(n-1)} \left( \sum_{i=1}^n i\right)^2 - \frac{1}{n(n-1)} \frac{n(n+1)(2n+1)}{6} - \frac{(n+1)^2}{4} \cr \cr 
&=& \frac{1}{n(n-1)} \frac{n^2(n+1)^2}{4} - \frac{(n+1)(2n+1)}{6(n-1)} - \frac{(n+1)^2}{4} \cr \cr 
&=& -\frac{n+1}{12} 
\label{eq:covxi}
\eeqn
after straightforward algebra.

We are now ready to calculate the expected value and variance of Wilcoxon rank sum. 
Let $x_1,x_2,\cdots,x_{n_A}$ be the ranks of the elements of group A, and 
$x_{n_A+1},x_{n_A+2},\cdots,x_N$ be the ranks of the elements of group B. 
In the absence of ties, $x_1, x_2,\cdots,x_N$ is a permutation of 
$1,2,3,\cdots,N$. The rank sum for group A is 
\[
  R_A = \sum_{i=1}^{n_A} x_i 
\]
\beq
  \Rightarrow \ \ \ E(R_A) = \sum_{i=1}^{n_A} E(x_i) = \sum_{i=1}^{n_A} \frac{N+1}{2} = \frac{n_A(N+1)}{2}\ \ \  \blacksquare
\eeq
The variance can be computed as follows. 
\beqn
  V(R_A) &=& V\left( \sum_{i=1}^{n_A} x_i \right) \cr \cr 
&=& \sum_{i=1}^{n_A} V(x_i) + \sum_{i\neq j} cov(x_i, x_j) \nonumber
\eeqn
Using~(\ref{eq:Vxi}) and (\ref{eq:covxi}), we have 
\beqn
  V(R_A) &=& \frac{n_A(N+1)}{12} - \frac{N+1}{12} \underbrace{\sum_{i \neq j} 1}_{=n_A(n_A-1)} \cr \cr \cr
  &=& \frac{n_A(N+1)}{12} - \frac{n_A(n_A-1)(N+1)}{12} \cr \cr 
&=& \frac{n_A(N+1)(N-n_A)}{12} \cr \cr 
&=& \frac{n_A n_B (N+1)}{12} \ \ \ \blacksquare
\eeqn

\section{Wilcoxon Rank Sum and U Statistic}

Let $R_A$ be the Wilcoxon Rank Sum for group A and $U_A$ be the U statistic for group A. Then 
\beq
  R_A = U_A + \frac{n_A(n_A+1)}{2} .
\eeq
The proof is very easy if there are no ties. It requires more algebra when there are ties. In the following, 
we first consider the case when there are no ties (not in group A at least). Then we tackle the general 
case when there are ties. 

{\bf Case 1}: No ties.

Sort the numbers in group A in ascending order: $A_1, A_2, \cdots, A_{n_A}$. Here $A_i$ ($i=1,2,\cdots,n_A$) 
denote the $i$th number in group A when sorted in ascending order. Let $B_1, B_2, \cdots, B_{n_B}$ be the 
numbers in group B sorted in ascending order. By no ties we mean that $A_i \neq A_j$ for all $i\neq j$ [$i, j \in 
(1,n_A)$] {\bf and} $A_i \neq B_j$ for all $i\in (1,n_A)$ and $j\in (1,n_B)$. In other words, all numbers in 
group A $A_1, A_2, \cdots, A_{n_A}$ are unequal {\bf and} none of the numbers in 
group B $B_1, B_2, \cdots, B_{n_B}$ is equal to any number in group A (However, there could be numbers 
in group B that are equal).

Let $r_i$ ($i=1,2,\cdots,n_A$) be the rank 
of $A_i$. Since there are no other numbers equal to $A_i$, $r_i$ is equal to one plus the number of 
numbers smaller than $A_i$. By 
definition, there are $i-1$ numbers in group A smaller than $A_i$ and $u_i$ numbers in group B smaller than 
$A_i$, where $u_i$ is the U count. Hence we have 
\beq
  r_i = i + u_i .
\label{eq:ri}
\eeq
Summing $i$ from $1$ to $n_A$ gives 
\beq
  R_A = \sum_{i=1}^{n_A} r_i = \sum_{i=1}^{n_A} i + \sum_{i=1}^{n_A} u_i = 
\frac{n_A(n_A+1)}{2} + U_A . \ \ \ \blacksquare
\eeq

{\bf Case 2}: There are ties, meaning that at least two numbers in 
$A_1, A_2, \cdots, A_{n_A}, B_1, B_2, \cdots, B_{n_B}$ are equal. 

There can also be multiple sets of ties. That is, at least two numbers are equal to a particular number, say $t_1$; 
at least two numbers are equal to another particular number, say $t_2$, ...etc. We can look at one particular set 
of ties, say $t_q$. Suppose there are $p$ numbers of $t_q$ occurring in groups A and B. Of these $p$ ties, $k$ of 
them are in group A and $p-k$ of them are in group B, where $k$ can be any integer between 0 and $p$. If $k=0$ (i.e.\ 
all the $p$ ties are in group B), we can disregard it because it has no effect on the rank sum $R_A$. So we focus on
the case when $k$ is between 1 and $p$. Suppose the $k$ ties in group A are $A_j, A_{j+1},\cdots 
A_{j+k-1}$. These $k$ ties have the same rank $r_j=r_{j+1}=\cdots = r_{j+k-1}\equiv r_{t_q}$ and the same 
U count $u_j=u_{j+1}=\cdots = u_{j+k-1}\equiv u_{t_q}$. From the definition of 
the rank in the case of ties, $r_{t_q}$ is equal to one plus the number of numbers smaller than $t_q$ plus $(p-1)/2$. 
Now there are $j-1$ numbers smaller than $t_q$ in group A. Let $u'_j$ be the number of numbers smaller than $t_q$ 
in group B. Then 
\beq
  r_j=r_{j+1}=\cdots = r_{j+k-1}=r_{t_q} = j + u'_j + \frac{p-1}{2} .
\label{eq:rtq}
\eeq
Recall that when there are ties, 1/2 is contributed to the U count for the numbers in group B that are equal to $t_q$. 
Since there are $p-k$ numbers in group B equal to $t_q$, the U count is given by 
\beq
  u_j=u_{j+1}=\cdots = u_{j+k-1} = u_{t_q} = u'_j + \frac{p-k}{2} .
\label{eq:uj}
\eeq
Combining equations~(\ref{eq:rtq}) and (\ref{eq:uj}) gives 
\beq
  r_j = r_{j+1}=\cdots = r_{j+k-1} = j + u_j - \frac{p-k}{2} + \frac{p-1}{2} = j+u_j+\frac{k-1}{2} .
\eeq
It follows that 
\beq
  \sum_{i=j}^{j+k-1} r_i = r_j+r_{j+1} + \cdots + r_{j+k-1} = kj + \frac{k(k-1)}{2} + k u_j .
\label{eq:sumties1}
\eeq
Since 
\beq
  \sum_{i=j}^{j+k-1} i = j + (j+1) + (j+2) + \cdots + (j+k-1) = kj + (1+2+\cdots k-1) = kj + \frac{k(k-1)}{2} ,
\eeq
we can write 
\beq
  \sum_{i=j}^{j+k-1} r_i = \sum_{i=j}^{j+k-1} i + \sum_{i=j}^{j+k-1} u_i = \sum_{i=j}^{j+k-1} (i+u_i) .
\eeq
This is the equation for a particular set of ties. The corresponding equations for the other sets of 
ties are equal to the equation above by changing $j$ and $k$ 
appropriate to the sets of ties. Summing over all ties appearing in group A, we have 
\beq
  \sum_{i \in {\rm ties}} r_i = \sum_{i \in {\rm ties}} (i+u_i) .
\eeq
When $A_i$ does not belong to any set of ties, equation~(\ref{eq:ri}) holds. Summing over numbers that don't 
belong to any set of ties, we have 
\beq
  \sum_{i \in {\rm no~ties}} r_i = \sum_{i \in {\rm no~ties}} (i+u_i) .
\eeq
Combining these two equations yield 
\beq
  \sum_{i=1}^{n_A} r_i = \sum_{i=1}^{n_A} (i + u_i) = \sum_{i=1}^{n_A} i +\sum_{i=1}^{n_A} u_i = \frac{n_A(n_A+1)}{2} 
+ U_A . \ \ \ \blacksquare 
\eeq

\section{Spearman's Rank-Order Correlation Coefficient} 

Let $x=(x_1,x_2,\cdots, x_n)$ and $y=(y_1,y_2,\cdots,y_n)$. If there are no ties in both $x$ and $y$, 
we can replace $x$ by a permutation of $(1,2,3,\cdots,n)$ and replace $y$ by another permutation 
of $(1,2,3,\cdots,n)$. If $x$ and $y$ are uncorrelated, we have $E(f(x_i) g(y_j))=E(f(x_i)) E(g(y_j))$, 
where $f$ and $g$ are arbitrary functions.

The Spearman's rank-order correlation coefficient is 
\beq
  r_s = \frac{1}{n} \sum_{i=1}^n Z_{x_i} Z_{y_i} ,
\eeq
where
\[
  Z_{x_i} = \frac{x_i - \bar{x}}{SD_x}  \ \ \ , \ \ \ Z_{y_i} = \frac{y_i - \bar{y}}{SD_y}
\]
are the Z-scores associated with $x$ and $y$. The mean and standard deviation of $x$ are
\beq
  \bar{x} = \frac{1}{n} \sum_{i=1}^n x_i  \ \ \ , \ \ \
SD_x = \sqrt{\frac{1}{n} \sum_{i=1}^n (x-x_i)^2} = \sqrt{\frac{1}{n} \sum_{i=1}^n x_i^2 - \bar{x}^2}
\eeq
Since $x$ is a permutation of $(1,2,3,\cdots,n)$,
\beq
  \bar{x} = \frac{1}{n} \sum_{i=1}^n i = \frac{n+1}{2} \ \ \ , \ \ \
  SD_x = \sqrt{\frac{1}{n} \sum_{i=1}^n i^2 - \frac{(n+1)^2}{4}} = \sqrt{\frac{n^2-1}{12}}
\eeq
using the results~(\ref{eq:arithmetic}) and (\ref{eq:squaresum}). Similarly, 
\beq
 \bar{y} = \bar{x} = \frac{n+1}{2} \ \ \ , \ \ \ SD_y=SD_x=\sqrt{\frac{n^2-1}{12}} .
\label{eq:meanssds}
\eeq
By construction, the expected value and variance of the Z-score are 
$E(Z_{x_i})=E(Z_{y_i})=0$ and $V(Z_{x_i})=V(Z_{y_i})=1$.

The expected value of $r_s$ is 
\[
  E(r_s) = E\left( \frac{1}{n} \sum_{i=1}^n Z_{x_i} Z_{y_i} \right) 
= \frac{1}{n} \sum_{i=1}^n E(Z_{x_i} Z_{y_i}) 
= \frac{1}{n} \sum_{i=1}^n E(Z_{x_i}) E(Z_{y_i}) = 0 \ \ \ \blacksquare
\]

The variance can be calculated as follows.
\beqn
   V(r_s) &=& \frac{1}{n^2} V \left( \sum_{i=1}^n Z_{x_i} Z_{y_i} \right) \cr \cr
 &=& \frac{1}{n^2} \left[ \sum_{i=1}^n V(Z_{x_i} Z_{y_i}) + \sum_{i\neq j} cov(Z_{x_i} Z_{y_i}, Z_{x_j} Z_{y_j}) 
\right]
\eeqn
\beqn
  V(Z_{x_i} Z_{y_i}) &=& E(Z_{x_i}^2 Z_{y_i}^2) - E^2(Z_{x_i} Z_{y_i}) \cr \cr 
&=& E(Z_{x_i}^2) E(Z_{y_i}^2) - \left[E(Z_{x_i}) E(Z_{y_i})\right]^2 \cr \cr 
&=& E(Z_{x_i}^2) E(Z_{y_i}^2) \nonumber 
\eeqn
It follows from $V(Z_{x_i})=V(Z_{y_i})=1$ and $E(Z_{x_i})=E(Z_{y_i})=0$ that 
$V(Z_{x_i})=E(Z_{x_i}^2)-E^2(Z_{x_i})=E(Z_{x_i}^2)$. So $E(Z_{x_i}^2)=E(Z_{y_i}^2)=1$ and 
$V(Z_{x_i} Z_{y_i})=1$. Thus, 
\beq
  V(r_s) = \frac{1}{n^2} \sum_{i=1}^n 1 + \frac{1}{n^2}\sum_{i\neq j} cov(Z_{x_i} Z_{y_i}, Z_{x_j} Z_{y_j}) 
= \frac{1}{n} + \frac{1}{n^2} \sum_{i\neq j} cov(Z_{x_i} Z_{y_i}, Z_{x_j} Z_{y_j}) 
\eeq
\beqn
  cov(Z_{x_i} Z_{y_i}, Z_{x_j} Z_{y_j}) &=& E(Z_{x_i} Z_{x_j} Z_{y_i} Z_{y_j}) 
- E(Z_{x_i} Z_{y_i}) E(Z_{x_j} Z_{y_j}) \cr \cr
&=& E(Z_{x_i} Z_{x_j}) E(Z_{y_i} Z_{y_j}) - E(Z_{x_i})E(Z_{y_i})E(Z_{x_j})E(Z_{y_j}) \cr \cr
&=& E(Z_{x_i} Z_{x_j}) E(Z_{y_i} Z_{y_j}) \cr \cr
&=& cov(Z_{x_i},Z_{x_j}) cov(Z_{y_i},Z_{y_j}) \cr \cr
&=& [ cov(Z_{x_i},Z_{x_j}) ]^2 
\eeqn
since $cov(Z_{x_i},Z_{x_j})=cov(Z_{y_i},Z_{y_j})$. 
\beqn
  cov(Z_{x_i},Z_{x_j}) &=& cov\left( \frac{x_i-\bar{x}}{SD_x}, \frac{x_j-\bar{x}}{SD_x} \right) \cr \cr
&=& \frac{1}{SD_x^2} cov(x_i,x_j) \cr \cr 
&=& -\frac{12}{n^2-1} \frac{n+1}{12} \cr \cr 
&=& -\frac{1}{n-1}
\eeqn
Hence, 
\beqn
  V(r_s) &=& \frac{1}{n} +\frac{1}{n^2} \frac{1}{(n-1)^2} \sum_{i \neq j} 1 \cr \cr 
  &=& \frac{1}{n} +\frac{1}{n^2} \frac{1}{(n-1)^2} n (n-1) \cr \cr 
  &=& \frac{1}{n} + \frac{1}{n(n-1)} \cr \cr 
  &=& \frac{1}{n} \left(1 + \frac{1}{n-1}\right) \cr \cr 
  &=& \frac{1}{n} \frac{n}{n-1} \cr \cr 
  &=& \frac{1}{n-1} \ \ \ \blacksquare
\eeqn

\end{document}
